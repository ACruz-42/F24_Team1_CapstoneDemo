% Gemini theme
% https://github.com/anishathalye/gemini

\documentclass[final]{beamer}

% ====================
% Packages
% ====================

\usepackage[T1]{fontenc}
\usepackage{lmodern}
\usepackage[size=custom,width=106.68,height=60.96,scale=0.85]{beamerposter}
\usetheme{custom}
\usecolortheme{tntech}
\usepackage{graphicx}
\usepackage{booktabs}
\usepackage{tikz}
\usepackage{pgfplots}
\usepackage{pgf}
\usepackage{multicol}
\usepackage{mathtools}

% ====================
% Lengths
% ====================

% If you have N columns, choose \sepwidth and \colwidth such that
% (N+1)*\sepwidth + N*\colwidth = \paperwidth
\newlength{\sepwidth}
\newlength{\colwidth}
\setlength{\sepwidth}{0.025\paperwidth}
\setlength{\colwidth}{0.3\paperwidth}

\newcommand{\separatorcolumn}{\begin{column}{\sepwidth}\end{column}}

\newcommand{\minus}{\scalebox{0.75}[1.0]{$-$}}
\newcommand{\smallequals}{\scalebox{0.75}[1.0]{$=$}}
\newcommand{\sectionHeading}[1]{\vskip2.0ex \textbf{#1} \vskip0.25ex}

% ====================
% Title
% ====================

\logotitle{\includegraphics[height=5.0cm]{tntechgold.png}}
\title{SECON 2025 Hardware Competition Robot}
\author{Sean Borchers, Alex Cruz, Samuel Hunter, and Dakota Moye}
\institute[shortinst]{Tennessee Technological University}

% ====================
% Body
% ====================

\pgfplotsset{compat=1.16}
\begin{document}

\begin{frame}[t]
\begin{columns}[t]
\separatorcolumn

\begin{column}{\colwidth}

  \begin{block}{Introduction}

    \sectionHeading{Problem}
    
        Build an autonomous robot for SECON.
    
    \sectionHeading{Constraints}
    
        Follow the rules

  \end{block}

  \begin{block}{Design}
    \begin{itemize}
    
      \item \textbf{Navigation}
        \begin{itemize}
          \item The robot navigates autonomously using orientation sensing.
        \end{itemize}
        
      \item \textbf{Control}
        \begin{itemize}
          \item Lower-level control controls motors.
        \end{itemize}
      
      \item \textbf{Camera}
        \begin{itemize}
          \item The camera detects objects and symbols.
        \end{itemize}
      
      \item \textbf{Sensors}
        \begin{itemize}
          \item Sensors detect orientation, magnetic fields, and light levels.
        \end{itemize}
    
    \end{itemize}
  
  \end{block}

    % \begin{figure}
    %   \centering
    %   \includegraphics[width=20.0cm]{Neuron.png}
    %   \caption{Neuron Diagram}
    % \end{figure}

    
    \begin{figure}
      \centering
      \includegraphics[width=20.0cm]{Mining Mayhem Field Opposite Corner.png}
      \caption{Game Field to compete on.}
    \end{figure}

    
    \begin{figure}
      \centering
      \includegraphics[width=20.0cm]{High_Block_Diagram.png}
      \caption{High Level Block Diagram.}
    \end{figure}


    
    % \begin{figure}
    %   \centering
    %   \includegraphics[width=20.0cm]{Mining Mayhem Field.png}
    %   \caption{Game Field to compete on.}
    % \end{figure}

\end{column}

\separatorcolumn

\begin{column}{\colwidth}

    \begin{figure}
      \centering
      \includegraphics[width=30.0cm]{Team Picture SECON 2025.jpg}
      \caption{Micah Rentschler, Sean Borchers, Caleb Sullivan, Alex Cruz, Cooper Nelson, Phoenix Sims, Sam Hunter, Dakota Moye, Nick Moulton}
    \end{figure}

    
    \begin{figure}
      \centering
      \includegraphics[width=20.0cm]{Robot_Practice_Field.jpg}
      \caption{Final Robot Product}
    \end{figure}
    
  \begin{block}{Experimentation}

    Robo does everything to a degree.

  \end{block}

\end{column}

\separatorcolumn

\begin{column}{\colwidth}

  \begin{block}{Conclusion}
    Put BOM and final thoughts here.

    % start example table

    % \begin{columns}[t]
    %   \begin{column}{0.4\colwidth}

    %     \begin{table}[ht]
    %       % increase table row spacing, adjust to taste
    %       \caption{Gen 1 ANN Model Assumptions}
    %       \label{Table:Gen1ANNAssumptions}
    %       \centering
    %       % Some packages, such as MDW tools, offer better commands for making tables
    %       % than the plain LaTeX2e tabular which is used here.
    %       \resizebox{\columnwidth}{!}{%
    %       \begin{tabular}{ c l }
    %         \toprule
    %         \textbf{Number} & \textbf{Assumption}\\
    %         \midrule
    %         1 & Firing Frequency Encoding\\

    %         2 & Steady State\\

    %         3 & Unity Static Firing Rate\\

    %         4 & Learned Inhibition\\

    %         5 & Unconditional Weighting\\
    %        \bottomrule
    %       \end{tabular}}
    %     \end{table}
    %   \end{column}

    %   \begin{column}{0.4\colwidth}
    %     \begin{table}[ht]
    %       % increase table row spacing, adjust to taste
    %       \caption{Gen 2 ANN Model Assumptions}
    %       \label{Table:Gen2ANNAssumptions}
    %       \centering
    %       % Some packages, such as MDW tools, offer better commands for making tables
    %       % than the plain LaTeX2e tabular which is used here.
    %       \resizebox{\columnwidth}{!}{%
    %       \begin{tabular}{ c l }
    %         \toprule
    %         \textbf{Number} & \textbf{Assumption}\\
    %         \midrule
    %         1 & Firing Frequency Encoding\\

    %         2 & Steady State\\

    %         3 & Learned Inhibition\\

    %         4 & Unconditional Weighting\\
    %         & \\
    %         \bottomrule
    %       \end{tabular}}
    %     \end{table}
    %   \end{column}

    % \end{columns}

    % end example table
    
  \end{block}

\end{column}

\separatorcolumn
\end{columns}
\end{frame}

\end{document}
